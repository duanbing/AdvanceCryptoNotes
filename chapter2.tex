 \documentclass[a4paper,11pt]{article}

 % define the title
\author{duanbing@baidu.com}
\title{SPDZ Protocol for Blockchain Privacy-Preserving Computing}
\usepackage{amsthm}
\usepackage{amsmath}
\usepackage{graphicx}
% margin control
\usepackage{geometry}
\geometry{a4paper,scale=0.8}

\usepackage{cite}

% flow control
\usepackage{tikz}
\usetikzlibrary{arrows,shapes,chains}

\newtheorem{myDef}{Definition}

\begin{document}

\maketitle  
\tableofcontents


\section{SPDZ Overview}
%From \cite{NIChooseAndCut}
%From \cite{Keller2018Overdrive}
  

\subsection{Pedersen Commitment on Secret share}

A commitment schema is cryptographic primitive that allows one to commit to a chosen value whole keep it hidden to others, with the ability to reveal the committed value later.

Pedersen Commitment is introduced firstly in \cite{pedersen1991non} aiming to verify someone(verifier) has received correct information about the secret without talking with other persons(prover). 
%https://crypto.stackexchange.com/questions/11923/difference-between-pedersen-commitment-and-commitment-based-on-elgamal


%https://whitepaper.io/document/63/enigma-whitepaper

%Practical Covertly Secure MPC for Dishonest Majority – or: Breaking the SPDZ Limits
% \cite{Ivan2013Practical}
 This algorithm are wildly adopted in secret hiding, auditing and equality verification.


%https://blog.csdn.net/turkeycock/article/details/97569082

%基于椭圆(EC)曲线的Pedersen承诺 https://zhuanlan.zhihu.com/p/108659500


\bibliographystyle{plain}
\bibliography{chapter2_ref} %这里的这个ref就是对文件ref.bib的引用

\end{document}